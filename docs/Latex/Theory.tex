% =========================================================================
% SciPost LaTeX template
% Version 1e (2017-10-31)
%
% Submissions to SciPost Journals should make use of this template.
%
% INSTRUCTIONS: simply look for the `TODO:' tokens and adapt your file.
%
% - please enable line numbers (package: lineno)
% - you should run LaTeX twice in order for the line numbers to appear
% =========================================================================


% TODO: uncomment ONE of the class declarations below
% If you are submitting a paper to SciPost Physics: uncomment next line
\documentclass[submission, Phys]{SciPost}
% If you are submitting a paper to SciPost Physics Lecture Notes: uncomment next line
%\documentclass[submission, LectureNotes]{SciPost}
% If you are submitting a paper to SciPost Physics Proceedings: uncomment next line
%\documentclass[submission, Proceedings]{SciPost}

\usepackage{braket}
\usepackage[ruled, vlined]{algorithm2e}
\usepackage{amsmath, bm}
\usepackage{dsfont}
\usepackage{pythonhighlight}
\SetKwComment{Comment}{$\triangleright$\ }{}

\begin{document}

\begin{center}{\Large \textbf{
			QuCumber: neural network-based generative modelling for quantum wavefunction reconstruction
		}}\end{center}

\begin{center}
	Matthew J.~S.~Beach,
	Isaac De Vlugt,
	Anna Golubeva,
	Patrick Huembeli$^\dag$,
	Roger G.~Melko\textsuperscript{*},
	Ejaaz Merali,
	Giacomo Torlai,
\end{center}

\begin{center}
	%{\bf 1} 
	Department of Physics and Astronomy, University of Waterloo,
	\\Ontario N2L 3G, Canada
	\\
	Perimeter Institute for Theoretical Physics, Waterloo,
	\\Ontario N2L 2Y5, Canada
	\\
	ICFO-Institut de Ciencies Fotoniques, Barcelona Institute of Science and Technology,
	\\08860 Castelldefels (Barcelona), Spain$^\dag$ \\
	* rgmelko@uwaterloo.ca \\
\end{center}

\begin{center}
	\today
\end{center}

% For convenience during refereeing: line numbers
%\linenumbers

\section*{Abstract}
{\bf
	In this post we present QuCumber, an open-source Python package which uses Restricted Boltzmann Machines for the reconstruction of quantum states from measurement data. 
    The use of modern machine learning techniques makes it possible to efficiently learn quantum states and access traditionally challenging many-body quantities which are not directly accessible to experiments. 
    We give examples of how to use QuCumber on positive-real and complex wavefunctions and how to extract meaningful observables such as energy, magnetization and fidelity.
}

\vspace{10pt}
\noindent\rule{\textwidth}{1pt}
\tableofcontents\thispagestyle{fancy}
\noindent\rule{\textwidth}{1pt}
\vspace{10pt}

\section{Introduction}

Scientific progress in controlled quantum systems has allowed for increasingly large-scale devices which can accurately prepare highly pure quantum states.
As the size of these devices grows to hundreds of qubits, we face the challenge of analyzing highly complex many-body states with a finite set of experimental measurements.
Since this computational power required scales exponentially with the number of qubits, there is a growing interest in using machine learning techniques which can successfully distill meaningful statistics from very large amounts of data.
In this post, we present a software package that can be used to characterize and analyze pure states produced from near-term 
quantum devices by {\it reconstructing} a quantum wavefunction with machine learning methods.

QuCumber is based on a Restricted Boltzmann Machines (RBM), an unsupervised generative modelling technique 
used widely in applications in machine learning~\cite{Smolensky}.
RBMs can represent many complicated objects including handwritten digit, natural images, and many-body wavefunctions~\cite{Torlai2016thermo, CarleoTroyer2017Science, ChenWang2018, GlasserCirac2018}.
An RBM can be interpreted as an approximate graphical representation of a probability distribution, which underlies 
a data set or physical system.
This distribution is defined by the energy of a classical Ising model, with tunable couplings which can be optimized during a training procedure.
With slight modifications, one can represent complex wavefunctions with an amplitude and phase~\cite{torlai2018tomography}.

The need for QuCumber arises from the growing availability of high quality data obtained from controlled quantum
systems and devices.
QuCumber provides a freely accessible open-source software package that supports wavefunction reconstruction directly from experimental data representing measurements of qubit eigenvalues, spin states, orbital occupation numbers, etc.
QuCumber is used to optimize the parameters of an RBM to model the most likely quantum state given a series of measurements,
a procedure analogous to the maximum likelihood technique, Bayesian inference or quantum state tomography. 
%however, our method does not scale exponentially with system size.
Once trained, the RBM parameters encode a compressed version of the underlying wavefunction.
New quantum state measurements can then be generated by the RBM, and observables that are otherwise not easily accessible from the original data can be calculated. 
%Examples of which include off-diagonal correlation functions and entanglement entropies~\cite{Torlai2016thermo, torlai2018tomography}. \textcolor{red}{this is not really a sentence...}
%QuCumber builds on previous studies which reconstructed wavefunctions from simulation and experimental data. 
Trained RBMs can be sampled to produce non-trivial measurements such as off-diagonal correlation functions and entanglement entropies~\cite{Torlai2016thermo, torlai2018tomography}.
In this paper, we describe how QuCumber can be used for these and other tasks in characterizing quantum many-body phenomenon on hundreds or thousands of qubits.
In the following, we provide code snippets written in Python 3, using PyTorch with CPU and GPU support~\cite{paszke2017automatic}.
We refer the reader to the full code documentation (\url{https://piquil.github.io/QuCumber/}) for further information on the code structure and operation.


% \subsection{State Reconstruction with RBMs}

% QuCumber allows to reconstruct the approximate wavefunction
% $\psi( \boldsymbol{\sigma} ) = \langle \boldsymbol{\sigma} | \psi \rangle$
% in some fixed basis
% $\{ \vert \boldsymbol{ \sigma} \rangle \}$
% from a set of experimental measurements.
% For instance, the data set could consist of a series of single qubit measurements
% $\vert {\boldsymbol{\sigma}} \rangle = \vert { \sigma}_1~{ \sigma}_2 \dots \rangle$
% in the computational basis obtained by preparing the state and measuring it several times.

% Any set of measurements on a quantum wavefunction is distributed according to the probability distribution defined by Born's rule:
% $p(\boldsymbol{\sigma}) = | \psi( \boldsymbol{\sigma} ) |^2$.
% The goal of QuCumber is to learn the best possible approximation to a wavefunction which underlies a set of measurement data.
% After the parameters of the RBM have been adjusted based on the data set, it becomes a compressed reconstruction of the original target wavefunction.
% New quantum states can now be generated by the RBM, and new observables measured.
% QuCumber implements a simple framework for sampling the reconstructed wavefuntion
% and performing measurements of physical observables on the generated samples.

In the simplest scenario the expansion coefficients of the wavefunction are all positive-real in the chosen basis.
Then, the wavefunction can be faithfully expressed as a probability distribution, and can be reconstructed 
from measurements in one basis using a standard RBM.  We discuss this case in the next section.
Note however, RBMs are not limited to positive-real wavefunctions, and QuCumber may also be used to reconstruct a 
wavefunction's full amplitude and phase, provided measurements are available multiple bases.
We discuss this case with a simple example in Section \ref{Sec:Training_QuCumber_on_complex_wavefunctions}.

Throughout this post, we provide a starting point for writing Python code employing QuCumber.
%However, there is much more theory to learn on the physics of wavefunction reconstruction with RBMs than present in this document.
For further background, the reader is urged to consult the relevant scientific literature, in particular Refs.~\cite{Torlai2016thermo, torlai2018tomography}.
A glossary of useful terms and equations appears at the end of the post in Section \ref{Glossary}.
In addition, extensive theoretical documentation and code tutorials are provided in the documentation for QuCumber at \url{https://piquil.github.io/QuCumber/}.


\section{Positive Real Wavefunctions}

In this section, we discuss the application of QuCumber to wavefunctions that are positive and real-valued in the computational basis.
That is, the expansion of the wavefunction has all positive and real coefficients in the basis in which experimental
measurements are performed.
Recall the Born rule, that the probability of measuring a certain quantum state is proportional to the square of its wavefunction.  
Then, the RBM has a simple encoding of the wavefunction as \cite{torlai2018tomography},
\begin{equation}
\psi_{\bm{\lambda}}= \sqrt{p_{\bm{\lambda}}}.
\end{equation}  
Here, ${\bm \lambda}$ are the RBM parameters, and $p_{\bm{\lambda}}$ is the marginal distribution (see the Glossary in Appendix \ref{Glossary}).
In this case, measurements in a single basis are sufficient for the reconstruction using fast learning techniques for RBMs such as contrastive divergence.
QuCumber has a highly efficient parallelizable method to determine the most likely state consistent with the given measurements.
As an example, we demonstrate how to train and sample an RBM model with measurements from the one-dimensional transverse-field Ising model.

\subsection{Transverse field Ising model}
\label{Sec:Training_TFIM}

The Hamiltonian for the transverse-field Ising model (TFIM) is given by
\begin{equation}
	\mathcal{H} = -J\sum_i \sigma^z_i \sigma^z_{i+1} - h \sum_i \sigma^x_i \label{TFIM}	
\end{equation}
where $\sigma^{\alpha}_i$ is a spin-1/2 Pauli operator on site $i$, with $\alpha=x,y,z$. We consider the critical point where $J=h=1$, which is the most difficult state to reconstuct.
For training data, we use a synthetic data set consisting of $M=10,000$ measurements in the $\sigma^z$-basis for $N=10$ spins (or qbits), generated with standard numerical techniques~\cite{itensor}.

The example dataset is provided in \url{https://github.com/PIQuIL/QuCumber/blob/master/examples/01_Ising/tfim1d_train_samples.txt}.
%\textcolor{red}{$N$ should be used for number of qubits, equal to number of visible units $V$.}

\subsection{An example of measurement data}
%\label{Sec:Training_TFIM}
To begin with wavefunction reconstruction we load the following packages.
\begin{python}
import torch
import numpy as np
import matplotlib.pyplot as plt
from qucumber.nn_states import PositiveWavefunction
from qucumber.callbacks import MetricEvaluator
import qucumber.utils.training_statistics as ts
import qucumber.utils.data as data
import quantum_ising_chain
from quantum_ising_chain import TFIMChainEnergy, TFIMChainMagnetization
\end{python}

Loading the tutorial data can be done as follows,

\begin{python}
train_samples_path = 'tfim1d_train_samples.txt'
psi_path           = 'tfim1d_psi.txt'

train_samples,target_psi = data.load_data(train_samples_path,psi_path)
\end{python}

Generally the input data needs to be in a numpy array or a torch tensor.
For a spin system with $N=10$ physical spins and measurements of every spin, one input data point will be an array of the form
\verb|np.array([1,0,1,1,0,1,0,0,0,1])|
, with shape \verb|(10,)|.
Here, we take the representation with $0$ denoting spin-down and $1$, spin-up (in the $z$-basis).
All the input data together must be an array of these arrays, which will have the shape \verb|(M,10)|, where $M$ is the number of data elements in the training set.

For the purpose of this tutorial, we will also compare the state reconstuction from QuCumber with the exact ground state.
This is also provided in \url{https://github.com/PIQuIL/QuCumber/blob/master/examples/01_Ising/tfim1d_psi.txt} and has already been loaded in \verb|target_psi|.

The central object of QuCumber is the representation of the wavefunction, which in the case of a positive-real wavefunction
is a standard RBM.
The Python object \verb|PositiveWavefunction| serves this purpose.
It posseses weights and biases ${\bm \lambda}$ which parameterize its properties.
To instanciate a positive wavefunction, one needs to specify the number of visible and hidden units in the RBM.
The number of visible units (\verb|num_visible|) is given by the size of the physical system $N$, i.e., the number of spins or qubits.
In contrast, the number of hidden units (\verb|num_hidden|) can be varied to change the expressiveness of the RBM.
Errors in the representation can be systematically improved by increasing the number of hidden units and consequently
the number of parameters (weights and biases) in the network.
Note, the quality of the reconstruction will depend on the specific wavefunction and the ratio $\alpha = \verb|num_hidden|/\verb|num_visible|$.
For some typical examples, we find that $\alpha = 1$ leads to good approximations of positive real wavefunctions \cite{Torlai2016thermo}.
In the general case, however, the value of $\alpha$ required for a given wavefunction reconstruction should be explored and adjusted by the user.

The following code snippet instanciates a \verb|PositiveWavefunction| object with 10 visible units (shape of the input data) and a neuron density $\alpha=1$ (\verb|nv=hv|).
The biases are initialized to zero, and the weights are initalized randomly according to a normal distrubution with zero mean and a variance of $1/$\verb|num_visible|.

\begin{python}
nv = train_samples.shape[-1]
nh = nv
nn_state = PositiveWavefunction(num_visible=nv,num_hidden=nh,gpu=False)
\end{python}

\subsection{Training an RBM}

The neural network wavefunction object can be trained with the constrastive divergence algorithm using various optimizers (see \url{https://pytorch.org/docs/stable/optim.html}).
This is implimented with the function \verb|PositiveWavefunction.fit|, which takes a number of hyperparameters, which we define the following way, 
\textcolor{red}{[cite specific documentation for the definition of these?]}

\begin{python}
epochs     = 1500
batch_size = 100
num_chains = 200
CD         = 10
lr         = 0.01
log_every  = 100
\end{python}
The choice of hyperparameter depends on the details of the system under study, as well as the size and quality of the training set.
In this example, we take a learning rate (\verb|lr|) of $10^{-2}$ using stochastic (batch) gradient descent with a batch size of 100, trained for 1,500 epochs.
The number of Gibbs sampling steps \verb|CD|, which dictates the number of steps in the contrastive divergence algorithm,
influences the speed of the training. The smaller we choose \verb|CD|, the faster is the training. High values on the other hand give a better approximation for the model distribution. It has been shown that even for \verb|CD|$=1$ the RBM trains well for many typical cases\cite{hinton2002training}. \verb|num_chain| defines the batch size of the negative phase of the RBM. The training shows better performance if this value is bigger than the batch size by a factor of 2 or 3. \textcolor{red}{Is this necessary in a tutorial. It is very specific and confusing.} With \verb|log_every| we define how often we calculate callbacks (See section~\ref{Sec:Callbacks}) during the training. Here we define that we calculate the callbacks every 100 epochs.
The success of training can be tracked by different measures, like the convergence of the energy or other observables.
In the example of the one-dimensional TIFM, we can compare fidelity of the true ground state with the trained model.
Note that this is only possible for very small system sizes, as the computation of the fidelity scale exponetially. Additionally we calculate the KL-divergence of the distribution of the training data and the distribution of the model.

\begin{python}
nn_state.space = nn_state.generate_hilbert_space(nv)

callbacks = [MetricEvaluator(log_every,{'Fidelity':ts.fidelity,'KL':ts.KL},target_psi=target_psi,verbose=True, 
                             space=nn_state.space)]
# The "verbose=True" argument will print the parameters in { } as a function of the training process.
\end{python}

The actual training is initialized by
\begin{python}
nn_state.fit(train_samples, epochs, batch_size, num_chains, CD, lr, progbar=False, callbacks=callbacks)
\end{python}
and outputs
\begin{python}
	Epoch = 50	Fidelity = 0.785383	KL = 0.441264
	Epoch = 100	Fidelity = 0.894514	KL = 0.208556
	Epoch = 150	Fidelity = 0.935645	KL = 0.125355
	Epoch = 200	Fidelity = 0.958977	KL = 0.080373
	Epoch = 250	Fidelity = 0.970116	KL = 0.058963
	Epoch = 300	Fidelity = 0.977806	KL = 0.043968
	Epoch = 350	Fidelity = 0.979086	KL = 0.041768
	Epoch = 400	Fidelity = 0.981559	KL = 0.037034
	Epoch = 450	Fidelity = 0.985201	KL = 0.029750
	Epoch = 500	Fidelity = 0.987439	KL = 0.025278
	Epoch = 550	Fidelity = 0.988868	KL = 0.022384
	Epoch = 600	Fidelity = 0.991312	KL = 0.017393
	Epoch = 650	Fidelity = 0.991913	KL = 0.016160
	Epoch = 700	Fidelity = 0.992833	KL = 0.014334
	Epoch = 750	Fidelity = 0.992946	KL = 0.014146
	Epoch = 800	Fidelity = 0.993429	KL = 0.013104
	Epoch = 850	Fidelity = 0.993416	KL = 0.013208
	Epoch = 900	Fidelity = 0.994565	KL = 0.010874
	Epoch = 950	Fidelity = 0.994806	KL = 0.010380
	Epoch = 1000	Fidelity = 0.995174	KL = 0.009641
\end{python}

After training is complete, the target distribution is encoded in the parameters of the RBM.
The quality of the approximate wavefunction reconstruction depends on many factors, such as the number of hidden units, size of the training set, or errors in the training procedure.
Evaluating and quantifying it is the task of the user for his or her given application; we outline some strategies in Section \ref{Sec:Callbacks}.

\subsubsection{Callbacks and monitoring training}
\label{Sec:Callbacks}

So far there is no certainty that the RBM performs well and indeed learns the probability distribution it is supposed to learn.
Therefore we introduce the concept of callbacks. 
To tract a certain observable like the energy or the fidelity during the training, one can implement it as a class in the \textbf{This should be outdated!!} \verb|observable.py| file and monitor it via callbacks.

\begin{python}
	Show example here
\end{python}

The user can track the training progress via these callbacks, for example by checking whether the energy converges to a reasonable value.
If the system size is not too big, one can even check whether the approximate RBM probability distribution has high fidelity with the theoretical state that should be learned.
\textbf{We could add plots with different learning behaviour to give a idea of bad learning rates}



\subsubsection{Monitoring fidelity for small systems}
The quantum state fidelity for pure states reduces to the overlap of two quantum states $\braket{\phi | \psi}$.
If one has access to the full quantum state $\ket{\phi} = \sum_{\bm{\sigma}} \phi(\bm{\sigma}) \ket{ \bm{\sigma}}$
and knows the coefficients $\phi(\bm{\sigma})$ that are approximated by the RBM with
$\psi_{\bm{\lambda}}(\bm{\sigma}) = \sqrt{p_{\bm{\lambda}}(\bm{\sigma})}$,
the fidelity can be computed via $\sum_{\bm{\sigma}} \phi(\bm{\sigma}) \psi_{\bm{\lambda}}(\bm{\sigma})$.
This can be used to compare the reconstructed wavefunction with the state underlying the measurement data, for example.
Because of the exponential growth of the wavefunction in the number of qubits, this is of course only feasible for small physical systems.
\begin{python}
    import utils.training_statistics as ts
    train_stats = ts.TrainingStatistics(train_samples.shape[-1],log_every)
\end{python}
%
To effectively calculate the fidelity with a target state $\ket{\phi}$, one needs to load the coefficients $\phi(\bm{\sigma})$. In this example we load the ground state of the TFIM Hamiltonian at the critical point $h = J = 1$ with the system size $N=10$.

\begin{python}
    N = 10 # Define system size
    phi = np.textload(...) # load wavefunction of TFIM
    train_stats.load_target_psi(phi)
    train_stats.fidelity(PositiveWavefunction)
    fidelity = train_stats.F
\end{python}

\textcolor{red}{[Let's load wavefunction of TFIM and calculate fidelity with the trained RBM from TFIM]}


\subsection{Saving and loading a model}
The parameters of a trained RBM can be saved and loaded to and from a file with

\begin{python}
	location = "some_folder/filename"
	nn_state.save(location)
	nn_state.load(location)
\end{python}

%\section{What can I do with a trained RBM?}

\subsection{Sampling from a RBM}
\label{Sec:Sampling_a-Trained_RBM}

RBMs are generative models, which means that they can produce new configurations of visible and hidden units
drawn according to the learned joint distribution.
These configurations are generated using Block Gibbs Sampling and represent a Markov chain.
The number of samples to generate (Gibbs steps) is specified by the variable $k$:
\begin{python}
	samples = nn_state.sample(samples=1000, k=10)
\end{python}

Drawing these samples is the main purpose of generative modelling. 
They can be used for a variety of tasks defined by the user.
For example, they can be used to calculate estimators for some physical observable.

\subsubsection{Observables}
\label{Sec:Observables}

Observables like energy, magnetization, or correlation functions can be calculated directly on the sampled data from the RBM.
For a general wavefunction $\Psi$ and a general observable (or operator) $\mathcal{O}$ the expectation value of this operator is given by
\begin{align}
	\braket{\mathcal{O}} & = \frac{\braket{\Psi | \mathcal{O} | \Psi}}{\braket{\Psi | \Psi}} = \frac{\sum_{\bm{\sigma}} \braket{\Psi | {\bm{\sigma}}} \braket{{\bm{\sigma}} | \mathcal{O} | \Psi}}{\sum_{\bm{\sigma}} \braket{\Psi | {\bm{\sigma}}} \braket{{\bm{\sigma}} | \Psi}} = \frac{\sum_{\bm{\sigma}} | \braket{\Psi | {\bm{\sigma}}} |^2 \frac{\braket{{\bm{\sigma}} | \mathcal{O} | \Psi}}{\braket{\Psi | {\bm{\sigma}}}} }{\sum_{\bm{\sigma}} | \braket{\Psi | {\bm{\sigma}}} |^2} \\
	                     & = \frac{\sum_{\bm{\sigma}} | \braket{\Psi | {\bm{\sigma}}} |^2 \mathcal{O}_L({\bm{\sigma}}) }{\sum_{\bm{\sigma}} | \braket{\Psi | {\bm{\sigma}}} |^2}
\end{align}
Here we have defined the local estimator,
\begin{equation}
	\mathcal{O}_L({\bm{\sigma}}) = \frac{\braket{{\bm{\sigma}} | \mathcal{O} | \Psi}}{\braket{\Psi | {\bm{\sigma}}}} .
\end{equation}
Considering that
\begin{equation}
	\mathcal{P}({\bm{\sigma}}) = \frac{ | \braket{\Psi | {\bm{\sigma}}} |^2 }{\sum_{\bm{\sigma}} | \braket{\Psi | {\bm{\sigma}}} |^2}
\end{equation}
is a probability distribution, the calculation of an expectation value of an operator $\mathcal{O}$ can be simplified
into the calculation of the average of the local random variable $\mathcal{O}_L({\bm{\sigma}})$ over the distribution $\mathcal{P}({\bm{\sigma}})$.
Thus, if we sample from the RBM using Gibbs sampling, where the probability distribution $\mathcal{P}({\bm{\sigma}})$ is the learned joint distribution,
we can evaluate $\braket{\mathcal{O}}$ by computing the mean of the random variable $\mathcal{O}_L({\bm{\sigma}})$ over the
sampled configurations ${\bm{\sigma}^{(k)}}$:
\begin{align}
	\label{Eq:}
	\mathcal{\langle O \rangle} \approx \frac{1}{N} \sum_{k=1}^N \mathcal{O}_L({\bm{\sigma}}^{(k)}).
\end{align}
\textcolor{red}{[CHECK definitions of $k$, $N$, and $M$ below... Also need to crosscheck with code. What it really returns.]}
Most observables depend on the physical system under study, e.g.~the Hamiltonian that underlies the original model, the lattice connectivity, etc.
Therefore, we now turn to a discussion of a specific model example. In Section~\ref{Sec:Training_TFIM} we trained an RBM on the TFIM data.
Now we will show how to calculate several quantities of interest from the learned distribution.


\subsubsection{Magnetization}

The magnetization of an Ising chain can be calculated simply by averaging over all Z measurements of the chain.
The RBM can only sample in the Z direction, therefore we have to average over the samples $\bm{\sigma}^{(k)}$.

\begin{python}
	magnetization = samples.mean()
\end{python}

\subsubsection{Energy using Observables module}
%\subsection{Example2: Energy of a TFIM chain}

For the transverse field Ising model a standard observable is the energy, obtained as the expectation value of
the Hamiltonian operator in equation~\ref{TFIM}.
Calculating the energy for a sample $\ket{\bm{\sigma} }= \ket{\sigma_1 \dots \sigma_n}$ is not as straightforward as for the magnetization,
because the Hamiltonian operator ${\sigma}^x_i$ is off-diagonal in the computational basis.
Just calculating $\bra{\bm{\sigma} } \mathcal{H} \ket{\bm{\sigma} }$ ignores the second part of the Hamiltonian completely.
%This is because the Hamiltonian is not a diagonal observable. 
If we write the wavefunction in its expanded form, $\ket{\psi} = \sum_{\bm{\sigma}} \psi(\bm{\sigma}) \ket{\bm{\sigma}} $,
we calculate the expectation value of the Hamiltonian using the local observable,
\begin{align}
	\braket{ \mathcal{H}}_L({\bm{\sigma}}) & = \sum_{ \bm{\sigma'}} \mathcal{H}_{\bm{\sigma}, \bm{\sigma}'} \cdot \psi(\bm{\sigma}') / \psi(\bm{\sigma})                                                     \\
	                             & =  \left[ -\sum_i \sigma_i \sigma_{i+1} - h \sum_i \frac{\psi(\sigma_1, \sigma_2, \dots , -\sigma_i, \sigma_{i+1} \dots)}{\psi (\bm{\sigma})} \right]
\end{align}
giving an estimator
\begin{equation}
	\braket{ \mathcal{H}} \approx \frac{1}{M} \sum_k^M \left[ -\sum_i \sigma_i \sigma_{i+1} - h \sum_i \frac{\psi (\bm{\sigma}_{-i})}{\psi (\bm{\sigma})} \right].
\end{equation}
Here we used that $\mathcal{H}_{\bm{\sigma}, \bm{\sigma}'} = \braket{\bm{\sigma} | H | \bm{\sigma}'} = -\sum_i \delta_{\bm{\sigma}, \bm{\sigma}'} - h \sum_i \delta_{\sigma_1, \sigma_1'} \dots \delta_{\sigma_i, -\sigma_i'} \dots$, and defined $\bm{\sigma}_{-i} = \sigma_1, \sigma_2, \dots , -\sigma_i, \sigma_{i+1} \dots$, which is the sample $\bm{\sigma}$ with the $i$-th spin flipped.
% The expectation value can be approximated by drawing $M$ samples and averaging over them.

The sum $\sum_i \sigma_i \sigma_{i+1}$ simply multiplies the neighbouring elements of the sample $\bm{\sigma}$. For the second sum one actually has to calculate the probabilities $\psi (\bm{\sigma})$ of the sample $\bm{\sigma}$ and $\bm{\sigma}_{-i} $ with
$\psi_{\bm{\lambda}}(\bm{\sigma}) = \sqrt{p_{\bm{\lambda}}(\bm{\sigma})}$
%\begin{equation}
%\psi(\bm{\sigma}) = \sqrt{p_{\lambda}(\bm{\sigma})}
%\end{equation}
with $p_{\lambda}(\bm{\sigma})$ from Eq.~\ref{Eq:marginal_distribution}. The partition functions $Z_{\lambda}$ cancel out if we divide the probabilities and the calculation is tractable also for large system sizes.
The \verb|observable.py| package contains a few examples like this.

\begin{python}
	sys.path.append('../examples/')
	from observables import quantum_ising_chain as TFIM
	# TFIM stands for transverse field ising model
	n_measurements= 50 # number of measurements
	hx=1.0       # Magnetic field
	tfim = TFIM.TransverseFieldIsingChain(hx,n_measurements)
	simulation = tfim.Run(nn_state,n_eq=200) # run

	Energy = simulation['energy']
\end{python}



\section{Complex wavefunctions}
\label{Sec:Training_QuCumber_on_complex_wavefunctions}

For the positive-real wavefunctions there is no additional information contained in the wavefunction
$\psi( \boldsymbol{\sigma})$ that could be lost by Borns rule $q(\boldsymbol{\sigma}) = | \psi( \boldsymbol{\sigma} ) |^2$.
For negative-real and complex functions the phase information contained in these coefficients will be lost if the measurement
is made in only one basis. Therefore, in experiments one has to apply local unitary transformations before the measurements
to capture the phase information. Given this additional information one can successfully perform quantum state reconstruction.
QuCumber also provides the possibility to do quantum state reconstruction for complex wavefunctions.

For complex wavefunctions, additionally to the marginal distribution (Eq.~\ref{Eq:marginal_distribution}) we require a phase factor
that has to be learned by an additional set of hidden units and corresponding network parameters $\bm{\mu}$ of the RBM:

\begin{align}
	\psi_{\bm{\lambda} \bm{\mu}} (\bm{\sigma})= \sqrt{p_{\bm{\lambda}} (\bm{\sigma})} e^{i \phi_{\bm{\mu}} (\bm{\sigma})/2},
\end{align}
%
where $\phi_{\bm{\mu}}(\bm{\sigma}) = \log (p_{\bm{\mu}} (\bm{\sigma}))$. The complex wavefunction can be rewritten in a generic basis $\{ \bm{\sigma}^b \}$as :

\begin{align}
	\psi_{\bm{\lambda} \bm{\mu}} (\bm{\sigma}^b)= \sum_{\bm{\sigma}} U (\bm{\sigma}^b, \bm{\sigma}) \psi_{\bm{\lambda} \bm{\mu}} (\bm{\sigma}),
\end{align}
%
with the unitary $U (\bm{\sigma}^b, \bm{\sigma})$ that rotates the state $\ket{\bm{\sigma}^b}$ to $\ket{\bm{\sigma}}$.

Our objective is to learn the full complex wavefunction $\psi_{\bm{\lambda} \bm{\mu}} (\bm{\sigma})$ in the reference basis (in this case, the computational basis) from the measurements of the rotated wavefunction $\psi_{\bm{\lambda} \bm{\mu}} (\bm{\sigma}^b)$.
The RBM again is trained by minimizing the negative log-likelihood (NLL). Following~\cite{} \textbf{ADD REFERENCE TO GIACOMOS THESIS OR SOMETHING THAT SHOWS THE MATH OF THE COMPLEX WAVEFUNCTION}, the gradients of the NLL will contain the unitaries $U (\bm{\sigma}^b, \bm{\sigma})$, and therefore we have to apply a slightly different learning algorithm.


\subsection{A two-qubit example}

In this section we work through a simple example for full quantum state tomography of a complex wavefunction with two qubits. 
First we load the packages needed,

\begin{python}
import torch
import numpy as np
import pickle

from qucumber.nn_states import ComplexWavefunction

from qucumber.callbacks import MetricEvaluator

import qucumber.utils.training_statistics as ts
import qucumber.utils.data as data              
import qucumber.utils.cplx as cplx              
import qucumber.utils.unitaries as unitaries
\end{python}


The data set comprises the measurements \verb|'qubits_train_samples.txt'|, the unitaries that have been applied before the measurement \verb|'qubits_train_bases.txt'| and the actual wavefunction $\ket{\psi}$ the measurements have been sampled from \verb|'qubits_psi.txt'|.

Load the files the following way:

\begin{python}
train_samples_path = 'qubits_train_samples.txt'
train_bases_path   = 'qubits_train_bases.txt'
bases_path         = 'qubits_bases.txt'
psi_path           = 'qubits_psi.txt'

train_samples,target_psi,train_bases,bases = data.load_data(train_samples_path, psi_path, train_bases_path, bases_path)
\end{python}


We will show in detail how to do tomography of the two-qubit state
$\ket{\psi} = 1/2 \{ \ket{00} - \ket{01} + \ket{10} - i \ket{11} \}$.
The state in \verb|target_psi| is with random coefficients and therefore not as suitable for a analytical demonstartion.
If we measure the state $\ket{\psi}$ in the computational basis Z, we obtain all qubit states ($\ket{00}$, $\ket{01}$, $\ket{10}$ and $\ket{11}$) with the same probability $p = 1/4$, but we do not learn anything about the phase $i$ or $(-1)$.
Therefore, we apply the local unitaries
\begin{align}
	H = \frac{1}{\sqrt{2}}
	\begin{bmatrix}
		1 & ~1 \\
		1 & -1 \\
	\end{bmatrix},~
	K = \frac{1}{\sqrt{2}}
	\begin{bmatrix}
		1 & -i \\
		1 & ~i \\
	\end{bmatrix}
\end{align}
on the state before we measure in the computational basis. Measuring the state in the computational basis is equivalent to applying the identity $\mathds{1}$ on the qubits, measuring in the $X$ basis is equivalent to applying a unitary $H$ on the qubit and then measure in $Z$ basis, measuring in $Y$ basis is equivalent to applying a $K$ and then measure in $Z$ basis. Because of the application of local unitaries the probability amplitudes of certain measurement outcomes start mixing and therefore we can extract more information about them. If one applies for example a Hadamard gate $H$ on the second qubit of the state $\ket{\psi}$, we obtain the state:

\begin{align}
	\mathds{1} \otimes H \ket{\psi} & = \frac{1}{2} \{ \ket{0+} - \ket{0-} + \ket{1+} - i \ket{1-} \}          \\
	                                & = \frac{1}{2 \sqrt{2}} \{ 2 \ket{01} + (1-i)\ket{10} + (1+i)\ket{11} \}.
\end{align}

If we measure this state in the Z basis, we obtain the state $\ket{01}$ with probability $p(\ket{01}) = 1/2$
and the other two states with $p(\ket{10}) = p(\ket{11}) = 1/4$.
The state $\ket{00}$ will never be measured. If we repeat the last step with a Hadamard gate on the first qubit, we obtain:
\begin{align}
	H \otimes \mathds{1} \ket{\psi} & = \frac{1}{2} \{ \ket{+0} - \ket{+1} + \ket{-0} - i \ket{-1} \}          \\
	                                & = \frac{1}{2 \sqrt{2}} \{ 2 \ket{00} + (1-i)\ket{01} - (1-i)\ket{11} \}.
\end{align}

Now $p(\ket{00}) = 1/2$, $p(\ket{01}) = p(\ket{11}) = 1/4$ and $p(\ket{10}) = 0$.
We can compare these results to an arbitrary two-qubit state $\ket{\psi}_c = c_1 \ket{00} +c_2 \ket{01} + c_3 \ket{10} +c_4 \ket{11}$,
with all the unitary transformations
\begin{align}
	\mathds{1} \otimes H \ket{\psi}_c & = \frac{1}{2\sqrt{2}} \{ (c_1+c_2) \ket{00} + (c_1 -c_2)\ket{01} + (c_3 + c_4)\ket{10} + (c_3-c_4)\ket{11} \}  \\
	H \otimes \mathds{1} \ket{\psi}_c & = \frac{1}{2\sqrt{2}} \{ (c_1+c_3) \ket{00} + (c_1 -c_3)\ket{01} + (c_2 + c_4)\ket{10} + (c_2-c_4)\ket{11} \}.
\end{align}

From the measurement without unitaries we know that all the probabilities are the same,
which means $c_i \in \{ \pm 1/2, \pm i/2 \}$.
From the application of $H$ to the second qubit we know that $c_1 = -c_2$ and $c_3 = \pm i c_4$.
From the application of $H$ to the first qubit we know that $c_1 = c_3$ and $c_2 = \pm i c_4$.
Already from this information we can do almost full tomography of the state $\ket{\psi}$.
If we fix $c_1 = 1/2$, which just defines a global phase, we find that $c_3 = 1/2$ and $c_2 = -1/2$.
We also find that $c_4 = \pm i/2$.
The only piece that is missing is the sign of $c_4$. To find this sign we apply the unitary $K$ on the second qubit:
\begin{align}
	\mathds{1} \otimes K \ket{\psi} & = \frac{1}{2} \{ \ket{0+} +i \ket{0-} + \ket{1+} - \ket{1-} \}           \\
	                                & = \frac{1}{2 \sqrt{2}} \{ 2 \ket{11} + (1-i)\ket{01} + (1+i)\ket{00} \}.
\end{align}
%
and compare it to the arbitrary state
%
\begin{align}
	\mathds{1} \otimes K \ket{\psi}_c & = \frac{1}{2\sqrt{2}} \{ (c_1-ic_2) \ket{00} + (c_1 +ic_2)\ket{01} + (c_3 -ic_4)\ket{10} + (c_3+ic_4)\ket{11} \},
\end{align}
%
where we find that $c_3 = ic_4$. Therefore, $c_4 = -i$.
Finding the full state $\ket{\psi}$ with this set of unitaries shows that it is a complete set of unitaries.
It might very well be that the complete set also contains the unitary $K \otimes \mathds{1}$,
if the amplitudes of the coefficients were not all the same ($|c_i|^2 \neq 1/4$).

The training set \verb|train_samples| was measured in the complete basis $\{ZZ, ZX, XZ, ZY, YZ \}$, and therefore \verb|train_bases| contains the unitaries applied to the respective measurement and has the form \verb|np.array([['Z','Z'], ['X','Z'], ['Z','X'], ...])|.
This means that the first measurement has been done in the Z basis for both qubits,
and therefore no unitary has been applied. The second measurement has been done in the XZ basis,
which means that the first qubit has been measured in the X basis. The training of the RBM is analogous to the case of positive-real wavefunction.
We define the training parameters,

\begin{python}
	epochs   = 200
	num_chains = 10
	batch_size = 10
	k     = 10
	lr     = 0.1
	log_every = 10
\end{python}

initialize the wavefunctions:

\begin{python}
nn_state.space = nn_state.generate_hilbert_space(nv) # generate the entire visible space of the system.
callbacks      = [MetricEvaluator(log_every,{'Fidelity':ts.fidelity,'KL':ts.KL},target_psi=target_psi,bases=bases,
                                  verbose=True, space=nn_state.space)]
# The "verbose=True" argument will print the parameters in { } as a function of the training process.
\end{python}

and start the training:

\begin{python}
nn_state.fit(train_samples, epochs, batch_size, num_chains, CD,
       lr, input_bases=train_bases, progbar=False, callbacks=callbacks)
\end{python}

After the training we can calculate state fidelity, observables or sample from the complex wavefunction
the same way we did from the real-positive wavefunction. However, one has to keep in mind that the sampling only works in the Z basis.

\subsection{Defining general unitaries}

QuCumber contains by default the unitaries $\mathds{1}$, $H$ and $K$, with:

\begin{align}
	\mathds{1} =
	\begin{bmatrix}
		1 & 0 \\
		0 & 1 \\
	\end{bmatrix},~
	H = \frac{1}{\sqrt{2}}
	\begin{bmatrix}
		1 & ~1 \\
		1 & -1 \\
	\end{bmatrix},~
	K = \frac{1}{\sqrt{2}}
	\begin{bmatrix}
		1 & -i \\
		1 & ~i \\
	\end{bmatrix}
\end{align}

New basis transformations can be added in the following way:

\begin{python}
	from qucumber import unitary
	import numpy as np
	new_unitary
	unitary_dict = unitaries.create_dict(name = 'new_unitary', unitary = new_unitary)
\end{python}

This creates a new unitary instance which is called \verb|'new_unitary'| and connects this name with the \verb|numpy| array
\verb|new_unitary| that has the form \verb|A = np.array([a,b],[c,d]])| for a matrix

\begin{align}
	A =
	\begin{bmatrix}
		a & b \\
		c & d \\
	\end{bmatrix}.
\end{align}

To call this basis transformation during the training, an element of the \verb|train_bases| array from above has to have the form
\verb|np.array(['Z','new_unitary'])| for a two-qubit example,
which means before the measurement the \verb|new_unitary| was applied to the second qubit.

So far we only used maximally one local unitary that is not the identity, such that maximally one qubit was not measured in the \verb|'Z'| basis.
QuCumber provides also the possibility for an arbitrary number \textcolor{red}{It is not really arbitrary, is there a maximum implemented} of non-trivial unitaries per measurement.
An element of \verb|train_bases|, for example, for a six-qubit state could have the following form:
\verb|np.array(['X','Z','new_unitary','Z','Y','X'])|

Apart from the training there is no difference in the positiv-real and the complex wavefunctions. The observables and callbacks are caculated as described in Section \ref{Sec:Sampling_a-Trained_RBM} and \ref{Sec:Callbacks}.
The sampling from the RBM is also exactly the same for both cases.
In the complex case, sampling from the RBM can only be performed in the original basis, which in our case was the Z basis.
\textcolor{red}{[what's correct? Double check this with code]}
%\section{Off diagonal observables in a trained complex wavefunction}
%
%We might add here an example for a non-trivial observable for complex wavefunctions.

\section{Conclusion}

We introduced the open-source package QuCumber for quantum state tomography with Restricted Boltzmann Machines and demonstrated on examples that
the package is applicable on positive-real and complex wavefunctions for any possible physical system with binary measurement outputs.
The class provided for the case of positive-real wavefunctions is highly parallelizable on GPUs and exhibits very high performance.
For complex wavefunctions QuCumber provides standard local unitaries for basis transformations
and can easily be equipped with customized unitaries, and therefore applied on any system with binary measurement outcomes.
After successful training of the RBM, one has full access to the wavefunction, respectively to the probability distribution of the measurements of the system.
One can sample new measurements, calculate observables or the fidelity to any other state.
We provide example code for any of these tasks to give the user a first impression of the implementation of

In the future QuCumber will be extended to the application of quantum state tomography on mixed states, multinomial quantum systems (like boson systems) and eventually on continuous variable systems.

\section*{Acknowledgements}
We thank G. Carleo, J. Carrasquilla and L. Hayward Sierens for stimulating discussions.
PIQuIL is supported by the Perimeter Institute for Theoretical Physics.

% TODO: include author contributions
\paragraph{Author contributions}
Authors are listed alphabetically. For an updated record of individual contributions, consult the repository at \url{https://github.com/PIQuIL/QuCumber/graphs/contributors}.

% TODO: include funding information
\paragraph{Funding information}
P.H. acknowledges support from ICFOstepstone - PhD Programme for Early-Stage Researchers in Photonics, funded by the Marie Sklodowska-Curie Co-funding of regional, national and international programmes (GA665884) of the European Commission, as well as by the Severo Ochoa 2016-2019' program at ICFO (SEV-2015-0522), funded by the Spanish Ministry of Economy, Industry, and Competitiveness (MINECO).
R.G.M. is supported in part by funding from the Natural Sciences and Engineering Research Council of Canada (NSERC) and a Canada Research Chair.
Research at Perimeter Institute is supported by the Government of Canada through Industry Canada and by the Province of Ontario through the Ministry of Research \& Innovation.


\appendix
\section{Glossary}
\label{Glossary}

We list an overview of terms discussed in the document and relevant for RBMs. For more detail we refer to the code documentation on \url{https://piquil.github.io/QuCumber/}, and References~\cite{hinton2002training, hinton2012practical}.

\begin{itemize}

	\item {\it Batch}: The subset of data selected for one iteration of training in stochastic gradient descent.

	\item {\it Biases}: For a visible unit $v_j$ and a hidden unit $h_i$, the respective biases in the RBM energy are $b_j$ and $c_i$. They act like a magnetic field term in the energy Eq~(\ref{RBMenergy}).

	\item {\it Contrastive Divergence}: An approximate maximum-likelihood learning algorithm for RBMs \cite{hinton2002training}.

	\item {\it Energy}: In analogy to statistical physics the energy of an RBM is defined given the joint configuration $(v,h)$ of visible and hidden units:
	      \begin{equation}
		      E_{\bm{\lambda}}(v,h) = - \sum\limits_{j=1}^V b_j v_j - \sum\limits_{i=1}^H c_i h_i - \sum\limits_{ij} h_i W_{ij} v_j, \label{RBMenergy}
	      \end{equation}

	\item {\it Effective energy}: Energy traced over the hidden units $h$:
	      \begin{equation}
		      \mathcal{E}_{\bm{\lambda}}(v) = - \sum\limits_{j=1}^V b_j v_j - \sum\limits_{i=1}^H \log \left\{ 1 + \exp \left( \sum\limits_{j} W_{ij}v_j +c_i\right) \right\}, \label{RBMeffectiveenergy}
	      \end{equation}

	\item {\it Epoch}: A single pass through an entire training set of data.

	\item {\it Hidden Units}: There are $H$ units in the second layer of the RBM, denoted by the vector $h=(h_1, ..., h_H)$, representing latent variables and are referred to as ``hidden". The number of hidden units $H$ can be adjusted to tune the representational capacity of the RBM.

	\item{\it Hyperparameter:} RBM architecture, and some details of training, that are fixed before training begins.  Examples are the learning rate, number of hidden units, batch size, or number of training epochs.

	\item {\it Joint distribution}: The RBM assigns a probability to each joint configuration $(v,h)$ according to the Boltzmann distribution of the energy,
	      \begin{equation}
		      p_{\bm{\lambda}}(v,h) = \frac{1}{Z_{\bm{\lambda}}} e^{-E_{\bm{\lambda}}(v,h)},
	      \end{equation}

\item{\it Learning Rate}: The learning rate is a hyperparamter that can be adjusted if the training performance is poor.
The lower the learning rate, the slower the training, but the more likely it is to find a good minimum of the objective function.
Large learning rates speed up the training and make it more likely to jump out of a local minimum, but they tend not to converge nicely.
Good practical values for the learning rates lie typically in the interval $[0.001, 0.1]$.


	\item {\it Marginal distribution}: Obtained by marginalizing the joint distribution, e.g.
	      \begin{equation}
		      \label{Eq:marginal_distribution}
		      p_{\bm{\lambda}}(v) = \frac{1}{Z_{\bm{\lambda}}} \sum\limits_{h\in \mathcal{H}} e^{-E_{\bm{\lambda}}(v,h)} = \frac{1}{Z_{\bm{\lambda}}} e^{- \mathcal{E}_{\bm{\lambda}}(v)}.
	      \end{equation}

	\item {\it QuCumber}: A quantum calculator used for many-body eigenstate reconstruction.

	\item {\it Parameters}: An RBM's energy is defined via a set of neural network parameters $\bm{\lambda} = \{b,c,W\}$, consisting of weights and biases.

	\item {\it Partition function}: The normalizing constant of the Boltzmann distribution. It is obtained by summing over all possible pairs of visible and hidden vectors,
	      \begin{equation}
		      Z_{\bm{\lambda}} = \sum\limits_{v\in \mathcal{V}}\sum\limits_{h\in \mathcal{H}} e^{-E_{\bm{\lambda}}(v,h)}.
	      \end{equation}

	\item {\it Restricted Boltzmann Machine}: A two-layer network with bidirectionally connected stochastic processing units. ``Restricted" refers to the connections (or weights) between the visible and hidden units. Each visible unit is connected with each hidden unit, but there are no intra-layer connections.

	\item {\it Visible Units}: There are $V$ units in the first layer of the RBM, denoted by the vector $v=(v_1, ..., v_V)$, which correspond to the experimental data and are therefore called ``visible". The number of visible units $V$ is fixed to the number of physical qubits.

	\item {\it Weights}: $W_{ij}$ is the symmetric connection or interaction between the visible unit $v_j$ and the hidden unit $h_i$.

\end{itemize}



	\section{Algorithm for a real positive wavefunction}
	The training algorithm of the RBM has the following structure.
	%=================================================================================
	% RBM.train

	\begin{algorithm}[H]
		\caption{Training Algorithm of QuantumReconstruction. \textbf{QR.train}() }
		\SetAlgoLined
		\For{batch in training set}{
			Load batch from training set, batch $=(\hat{ \boldsymbol{\sigma}}_1~\hat{ \boldsymbol{ \sigma}}_2 \dots)$\;
			compute the gradients from the batch $\Delta \bm{\Theta } =$ ($\Delta W$, $\Delta b$, $\Delta c$)\\
			\textbf{\lstinline{QR.compute_batch_gradients}}(k, batch, basis)  \Comment*[r]{Algorithm 2}
			update weights and biases \\
			$\bm{\Theta} \leftarrow \bm{\Theta } - \Delta \bm{\Theta } $ \;
		}

	\end{algorithm}

	%=================================================================================
	% RBM.compute gradient

	The gradients are calculated according to the contrastive divergence algorithm, which allows us to approximate the probability distribution of the model with $k$ Gibbs sampling steps from the actual training data.

	\begin{algorithm}[H]
		\caption{Compute Gradient from Batch. \textbf{\lstinline{QR.compute_batch_gradients}}(k, batch, basis) }
		\SetAlgoLined
		\uIf{basis = None}{
			Reset gradients $\Delta W$, $\Delta h_b$, $\Delta v_b$ $= 0$\;
			\For{$\hat{ \boldsymbol{ \sigma}}_i$ in batch}{
				sample $\bm{h}_0$, $\bm{v}_k$ and $\bm{p}_{h_k}$ from $\hat{ \boldsymbol{ \sigma}}_i$\\
				\textbf{\lstinline{RBM.gibbs_sampling}}(k, $\hat{ \boldsymbol{ \sigma}}_i$)  \Comment*[r]{Algorithm 3}
					calculate gradients\\
				$\Delta W += \bm{v}_0 \bm{h}_0^T - \bm{v}_k \bm{p}_{h_k}^T$ \\
				$\Delta c += \bm{h}_0 - \bm{p}_{h_k}$ \\
				$\Delta b += \bm{v}_0 - \bm{v}_k$ \;
			}
			$M = \vert batch \vert$ \;
			return $\Delta W / M$, $\Delta c / M$, $\Delta b / M$ \;}
		\Else{Do complex gradient \Comment*[r]{Algorithm 6}}
	\end{algorithm}

	%=================================================================================
	% RBM.gibbs sampling

	The Gibbs sampling is done $k$ times back and forth. The contrastive divergence algorithm already shows good results for $k=1$. But for better results this value can be increased.

	\begin{algorithm}[H]
		\caption{Gibbs sampling. \textbf{\lstinline{RBM.gibbs_sampling}}(k, $\hat{ \boldsymbol{ \sigma}}_i$) }
		\SetAlgoLined
		calculate $\bm{p}_h$ and sample $\bm{h}_0$ from $\hat{ \boldsymbol{ \sigma}}_i$\\
		\textbf{\lstinline{RBM.sample_h_given_v}}($\hat{ \boldsymbol{ \sigma}}$)  \Comment*[r]{Algorithm 4}
		$\bm{h} = \bm{h}_0$\;
			i = 0\;
			\While{i $\leq$ k}{
				calculate $\bm{p}(\bm{v}|\bm{h}_i)$ and sample $\bm{v}$ from $\bm{h}$\\
				\textbf{\lstinline{RBM.sample_v_given_h}}($\bm{h}$)  \Comment*[r]{Algorithm 5}
					calculate $\bm{p}(\bm{h}|\bm{v})$ and sample $\bm{h}$ from $\bm{v}$\\
					\textbf{\lstinline{RBM.sample_h_given_v}}($\bm{v}$)  \Comment*[r]{Algorithm 4}
				$i +=1$
			}
			return $\bm{p}_{h_k} =$ $\bm{p}(\bm{h}|\bm{v})$, $\bm{v}_k = \bm{v}$ and $\bm{h}_0$\;

	\end{algorithm}


	%=================================================================================
	% RBM.v given h and vice versa



	\begin{algorithm}[H]
		\caption{calculate $\bm{p}(\bm{v}|\bm{h})$ and sample $\bm{v}$ \textbf{\lstinline{RBM.v_given_h}}() }
		\SetAlgoLined
		calculate probability $\bm{p}(\bm{v} = 1|\bm{h}) = \sigma( W \bm{h} + v_b)$\;
		Bernoulli sample $\bm{v}$ from this probability\;
		return $\bm{v}$ and $\bm{p}(\bm{v} = 1|\bm{h})$\;

	\end{algorithm}


	\begin{algorithm}[H]
		\caption{calculate $\bm{p}(\bm{h}|\bm{v})$ and sample $\bm{h}$ \textbf{\lstinline{RBM.h_given_v}}() }
		\SetAlgoLined
		calculate probability $\bm{p}(\bm{h} = 1|\bm{v}) = \sigma(\bm{v}^T W + h_b)$\;
		Bernoulli sample $\bm{h}$ from this probability\;
		return $\bm{h}$ and $\bm{p}(\bm{h} = 1|\bm{v})$\;

	\end{algorithm}

% TODO:
% Provide your bibliography here. You have two options:

% FIRST OPTION - write your entries here directly, following the example below, including Author(s), Title, Journal Ref. with year in parentheses at the end, followed by the DOI number.
%\begin{thebibliography}{99}
%\bibitem{1931_Bethe_ZP_71} H. A. Bethe, {\it Zur Theorie der Metalle. i. Eigenwerte und Eigenfunktionen der linearen Atomkette}, Zeit. f{\"u}r Phys. {\bf 71}, 205 (1931), \doi{10.1007\%2FBF01341708}.
%\bibitem{arXiv:1108.2700} P. Ginsparg, {\it It was twenty years ago today... }, \url{http://arxiv.org/abs/1108.2700}.
%\end{thebibliography}

% SECOND OPTION:
% Use your bibtex library
% \bibliographystyle{SciPost_bibstyle} % Include this style file here only if you are not using our template
\bibliography{bibliography}

\nolinenumbers

\end{document}
